\documentclass{article}
\usepackage[utf8]{inputenc}
\usepackage{amsmath}
\usepackage{amssymb}
\begin{document}
\textbf{Odometry model}\\\\
$\Delta_{l}$ = $Ticks_{l}$ - $Previousticks_{l}$\\\\
$\Delta_{r}$ = $Ticks_{r}$ - $Previousticks_{r}$\\\\
The wheels rotate at 720 ticks per revolution\\\\
Distance per tick = $\frac{\text{2$\pi$r}}{720}$\\\\
$V_{l}$ = $\frac{\text{$\Delta_{l}$ x Distance per tick}}{dt}$\\\\
$V_{r}$ = $\frac{\text{$\Delta_{r}$ x Distance per tick}}{dt}$\\\\
$V_{x_{cm}}$ = $\frac{\text{$V_{l}$ + $V_{l}$}}{2}$\\\\
Assuming that the robot does not slip along a direction perpendicular to the direction of motion\\
$V_{y_{cm}}$ = 0\\\\
By fixing one wheel at rest at a time and making the robot rotate centered at each wheel we get the rotational velocity of the robot as\\\\
$\omega$ = $\frac{\text{$V_{r}$-$V_{r}$}}{l}$, where l is the distance between two wheels\\\\
$\Delta_{x}$ = $V_{x_{cm}}$cos$(\theta)$dt  \\\\
$\Delta_{y}$ = $V_{x_{cm}}$sin$(\theta)$dt \\\\
$\Delta_{\theta}$ = $\omega$dt\\\\
$x_{t}$ = $x_{t-1}$ + $\Delta_{x}$\\\\
$y_{t}$ = $y_{t-1}$ + $\Delta_{y}$\\\\
$\theta_{t}$ = $\theta_{t-1}$ + $\Delta_{\theta}$\\\\
\end{document}